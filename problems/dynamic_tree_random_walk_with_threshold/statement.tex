\documentclass{article}
\usepackage{amsmath, amssymb, fullpage}
\begin{document}

\section*{Problem Description}
In the mystical forest of Arboria, there stands an ancient tree of shrines. Each shrine~$i$ harbors a latent energy, denoted by $a_i$. Pilgrims begin their journey at a chosen shrine~$s$. At each shrine they visit, if the energy there exceeds a certain tolerance~$T$, the pilgrim becomes overwhelmed and ends their journey. Otherwise, the pilgrim proceeds to one of the direct child‑shrines chosen uniformly at random; if there are no child‑shrines, the journey ends naturally.

The tree’s energy is not static – from time to time, rituals alter the energy of a single shrine. Your task is to assist the pilgrims by answering their queries: given a starting shrine~$s$ and a tolerance~$T$, what is the expected number of shrines they will visit before their journey ends?

Formally, you are given a rooted tree with $n$ nodes (the root is node~$1$). Each node $i$ has a value $a_i$. Consider a random walk starting at node $s$: at each step, if the current node's value is greater than a given threshold $T$, the walk stops. Otherwise, the walk moves to a uniformly random child (if there are no children, it stops). You need to process two types of operations:
\begin{enumerate}
\item Update: change the value of a node $x$ to $v$.
\item Query: given $s$ and $T$, output the expected number of steps until the walk stops.
\end{enumerate}

\section*{Input}
The first line contains an integer $n$ ($1 \le n \le 200,000$). \\
The second line contains $n$ integers $a_1, a_2, \dots, a_n$ ($1 \le a_i \le 10^9$). \\
The next $n-1$ lines each contain two integers $u$ and $v$ ($1 \le u, v \le n$), meaning $v$ is a child of $u$. The edges form a tree rooted at $1$. \\
The next line contains an integer $q$ ($1 \le q \le 200,000$). \\
Then $q$ lines follow, each describing an operation:
\begin{itemize}
\item \texttt{1 x v} ($1 \le x \le n$, $1 \le v \le 10^9$): update $a_x$ to $v$.
\item \texttt{2 s T} ($1 \le s \le n$, $1 \le T \le 10^9$): query the expected steps starting from $s$ with threshold $T$.
\end{itemize}

\section*{Output}
For each query of type 2, output the expected number of steps. Your answer is considered correct if the absolute or relative error is less than $10^{-6}$.

\section*{Constraints}
\begin{itemize}
\item $n \le 200,000$
\item $q \le 200,000$
\item $1 \le a_i, v, T \le 10^9$
\item Time limit: 2 seconds
\item Memory limit: 256 MB
\end{itemize}

\section*{Sample}
Input:
\begin{verbatim}
3
1 2 3
1 2
2 3
5
2 1 5
1 3 10
2 1 5
1 2 10
2 1 5
\end{verbatim}
Output:
\begin{verbatim}
2.0000000000
2.0000000000
1.0000000000
\end{verbatim}

\noindent
\textbf{Explanation:}
The tree is a chain $1 \to 2 \to 3$. Initially, $a = [1,2,3]$.
\begin{itemize}
\item First query: $s=1, T=5$. The walk always goes $1 \to 2 \to 3$ and stops at $3$ (no children). Hence, exactly $2$ steps are taken.
\item Update: set $a_3 = 10$.
\item Second query: $s=1, T=5$. The walk still goes $1 \to 2 \to 3$, but at node $3$ the energy $10 > 5$ causes an immediate stop. The number of steps remains $2$.
\item Update: set $a_2 = 10$.
\item Third query: $s=1, T=5$. Now at node $2$ the energy $10 > 5$, so the walk stops after the first step from $1$ to $2$. Thus, the expected number of steps is $1$.
\end{itemize}

\end{document}