\documentclass[12pt,a4paper]{article}
\usepackage{amsmath}
\usepackage{amssymb}
\usepackage{geometry}
\geometry{margin=1in}
\begin{document}

\begin{center}
    \Large\textbf{Tree Path Maximum with Vertex Updates}
\end{center}

\textbf{Time Limit:} 2 seconds \\
\textbf{Memory Limit:} 256 MB

\section*{Problem Description}
You are given a rooted tree with $n$ vertices numbered from $1$ to $n$. Each vertex $v$ has an integer value $a_v$. You need to process $q$ queries of two types:

\begin{enumerate}
    \item \texttt{1 v x}: Update the value of vertex $v$ to $x$.
    \item \texttt{2 u v}: Consider the simple path from $u$ to $v$ in the tree. You want to select a subset of vertices from this path such that no two selected vertices are adjacent along the path. Find the maximum possible sum of values of the selected vertices.
\end{enumerate}

The tree structure remains static (no edges are added or removed), but vertex values can be updated dynamically. You must process each query online in the order they are given.

\section*{Input Format}
The first line contains two integers $n$ and $q$ $(1 \le n, q \le 200\,000)$.

The second line contains $n$ integers $a_1, a_2, \dots, a_n$ $(-10^9 \le a_i \le 10^9)$.

The next $n-1$ lines each contain two integers $u$ and $v$ $(1 \le u, v \le n)$, denoting an edge between vertices $u$ and $v$. It is guaranteed that the edges form a tree.

The next $q$ lines describe the queries. Each query is in one of the following formats:
\begin{itemize}
    \item \texttt{1 v x} $(1 \le v \le n$, $-10^9 \le x \le 10^9)$
    \item \texttt{2 u v} $(1 \le u, v \le n)$
\end{itemize}

\section*{Output Format}
For each query of type $2$, output a single integer: the maximum possible sum of a subset of vertices on the path from $u$ to $v$ with no two selected vertices being adjacent along the path.

\section*{Constraints}
\begin{itemize}
    \item $1 \le n, q \le 200\,000$
    \item $-10^9 \le a_i, x \le 10^9$
    \item The graph is a tree with vertices numbered from $1$ to $n$.
    \item Queries are processed online in the given order.
\end{itemize}

\section*{Sample Input}
\begin{verbatim}
5 5
1 -2 3 -4 5
1 2
2 3
3 4
3 5
2 1 4
1 3 10
2 1 4
1 2 7
2 2 5
\end{verbatim}

\section*{Sample Output}
\begin{verbatim}
3
10
12
\end{verbatim}

\section*{Note}
In the first query, the path from $1$ to $4$ is $1 \to 2 \to 3 \to 4$ with values $[1, -2, 3, -4]$. The optimal selection is $\{1, 3\}$ (sum $4$) or $\{3\}$ (sum $3$) if we cannot take adjacent vertices. Since $1$ and $3$ are two steps apart, they are not adjacent along the path. Wait, let's check carefully: In the path order, vertices $1$ and $3$ are separated by vertex $2$, so they are not adjacent. However, we must ensure no two selected vertices are adjacent in the path. The subset $\{1, 3\}$ is valid because they are not adjacent (vertex $2$ is between them). The sum is $1+3=4$. But the sample output says $3$. Let's recalculate: The path is $1,2,3,4$. Possible valid subsets: 
- $\{1,3\}$: sum $4$, but are $1$ and $3$ adjacent? No, because the path order is 1-2-3, so 1 and 3 are not adjacent.
- $\{1,4\}$: sum $-3$, and they are not adjacent? Path: 1-2-3-4, so 1 and 4 are not adjacent.
- $\{2,4\}$: sum $-6$, not adjacent? 2 and 4 are separated by 3, so valid.
- $\{1\}$: sum $1$
- $\{2\}$: sum $-2$
- $\{3\}$: sum $3$
- $\{4\}$: sum $-4$
- $\{1,4\}$: sum $-3$
- $\{2,4\}$: sum $-6$
- $\{1,3\}$: sum $4$ ← maximum.
But the sample output says $3$. There must be a mistake. Possibly the intended interpretation is that the selected vertices must not be adjacent in the tree? No, the problem says "adjacent in the path". Alternatively, maybe the root matters? The problem says rooted tree but doesn't specify the root. Let's assume the root is 1. Then the path from 1 to 4 is indeed 1-2-3-4. So the maximum should be 4. However, the sample output is 3. Let's check the values: 1, -2, 3, -4. If we take {3} only, sum=3. If we take {1,3}, sum=4. But 1 and 3 are not adjacent in the path? They are separated by 2, so they are not adjacent. So 4 should be possible. Wait, maybe the problem considers the vertices as adjacent if they are consecutive in the path sequence. In the path 1-2-3-4, vertices 1 and 3 are not consecutive, so they are not adjacent. So I think the sample output might be wrong? Or perhaps I misread the problem: "no two selected vertices are adjacent in the path" means if the path is v1, v2, ..., vk, then we cannot select both vi and vi+1 for any i. So in the sequence [1,2,3,4], we cannot select both 1 and 2, or 2 and 3, or 3 and 4. But we can select 1 and 3 because they are not consecutive. So {1,3} is valid. So sum=4. But the sample output is 3. Let's reexamine the sample input and output.

Maybe the tree is not rooted at 1? The problem says "rooted tree" but doesn't specify the root. In many tree problems, the root is often 1 unless specified. But here, the edges are given as undirected, and the problem says "rooted", which might imply that the tree is rooted at 1. However, the path between two vertices is the simple path in the undirected tree, so the root doesn't affect the path.

Perhaps the issue is that the values can be negative. In the subset selection, we are not required to take vertices, so we can skip negative ones. In the path [1,2,3,4] with values [1,-2,3,-4], the best is {1,3}=4. So why would the output be 3? Could it be that we are not allowed to take non-adjacent vertices that are two steps apart? No, the condition is only about adjacency.

Maybe the sample output is incorrect in this initial draft. Let's adjust the sample to match the correct logic.

Let's create a new sample that is consistent.

Consider a tree:
n=4, q=3
values: 5, -1, 3, 2
edges: 1-2, 2-3, 2-4
queries:
2 1 3 → path 1-2-3, values [5,-1,3]. Valid subsets: {5,3}=8 (1 and 3 are not adjacent in the path? In path 1-2-3, 1 and 3 are not adjacent because they are separated by 2, so allowed), {5}=5, {3}=3, {-1}=-1, {5,3}=8 is max.
1 2 4 → update vertex 2 to 4.
2 1 3 → path 1-2-3, values [5,4,3]. Now {5,3}=8, {5,4}=9 (but 5 and 4 are adjacent? In the path, 5 (vertex1) and 4 (vertex2) are consecutive, so not allowed), {4,3}=7 (adjacent? 4 and 3 are consecutive, not allowed), {5,3}=8 is allowed. So output 8.
2 3 4 → path 3-2-4, values [3,4,2]. Valid subsets: {3,2}=5 (not adjacent, since 3 and 2 are consecutive? Wait, path: 3,2,4. 3 and 2 are consecutive, so cannot take both. Actually, in the path, vertices are in order: 3,2,4. So 3 and 2 are adjacent, so cannot take both. 3 and 4 are not adjacent (separated by 2), so {3,4}=7 is allowed. Also {2,4}=6 (2 and 4 are not adjacent? In path 3-2-4, 2 and 4 are consecutive? The path is 3,2,4 so 2 and 4 are consecutive? Yes, because the sequence is 3,2,4 so 2 and 4 are adjacent. So cannot take both. So the only valid subsets with two elements are {3,4}=7. Also single elements: {3}=3, {4}=4, {2}=2. So max is 7.

So sample input:
4 3
5 -1 3 2
1 2
2 3
2 4
2 1 3
1 2 4
2 1 3
2 3 4

Output:
8
8
7

But this might be too trivial. Let's use the original sample but adjust values to make the output correct.

Alternatively, let's stick with the original sample but change the output to match the correct computation. But the problem statement should be consistent.

Given the confusion, I will adjust the sample input and output so that they are consistent with the problem description.

Let's create a sample where the correct answer is clear.

Sample Input:
5 4
1 2 3 4 5
1 2
1 3
2 4
2 5
2 4 5
1 2 -10
2 4 5
2 1 3

Explanation:
Tree:
    1
   / \
  2   3
 / \
4   5

Initial values: a1=1, a2=2, a3=3, a4=4, a5=5.
First query: path from 4 to 5: 4-2-5. Values [4,2,5]. Valid subsets: {4,5}=9 (not adjacent in path), {4}=4, {2}=2, {5}=5, {4,2} not allowed (adjacent), {2,5} not allowed. Max=9.
Second query: update vertex 2 to -10.
Third query: path 4-2-5: values [4,-10,5]. Valid subsets: {4,5}=9, {4}=4, {-10}=-10, {5}=5. Max=9.
Fourth query: path 1-3: 1-3 (direct). Values [1,3]. Valid subsets: {1,3}=4 (not adjacent? In path 1-3, they are adjacent because the path has only two vertices? Actually, the path is 1 and 3, they are adjacent in the path because they are consecutive. So we cannot take both. So we can take only one: max(1,3)=3.

So output:
9
9
3

This is consistent.

So I will use this sample.

Thus, the sample input and output in the problem statement should be:

Sample Input:
5 4
1 2 3 4 5
1 2
1 3
2 4
2 5
2 4 5
1 2 -10
2 4 5
2 1 3

Sample Output:
9
9
3

Now, let's write the problem statement with this sample.

\end{verbatim}
```

\section*{Sample Input}
\begin{verbatim}
5 4
1 2 3 4 5
1 2
1 3
2 4
2 5
2 4 5
1 2 -10
2 4 5
2 1 3
\end{verbatim}

\section*{Sample Output}
\begin{verbatim}
9
9
3
\end{verbatim}