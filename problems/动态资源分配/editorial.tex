\documentclass{article}
\usepackage{amsmath, amssymb, fullpage}
\usepackage{algorithm}
\usepackage{algpseudocode}
\usepackage{enumitem}

\begin{document}

\title{Editorial: 动态资源分配}
\author{}
\date{}
\maketitle

\section*{问题简述}
有 $m$ 种资源和 $n$ 个项目。每个资源 $j$ 的总量为 $C_j$,每个项目 $i$ 需要固定的资源集合 $S_i \subseteq \{1,\dots,m\}$。在 $q$ 个时间点上,每个项目 $i$ 有需求量 $D_{t,i}$。对于每个时间点 $t$,判断是否存在一种资源分配方案,使得:
\begin{enumerate}[itemsep=0pt]
    \item 每个资源 $j$ 分配的总量不超过 $C_j$。
    \item 每个项目 $i$ 获得的总资源量等于 $D_{t,i}$。
    \item 资源只能分配给需要它的项目。
\end{enumerate}

\section*{核心观察}
\subsection*{可行性转化为最大流}
将原问题建模为网络流:
\begin{itemize}[itemsep=0pt]
    \item 源点 $S$ 向每个资源点 $j$ 连边,容量为 $C_j$。
    \item 每个资源点 $j$ 向所有需要它的项目点 $i$(即 $j \in S_i$)连边,容量为 $+\infty$(或足够大的值)。
    \item 每个项目点 $i$ 向汇点 $T$ 连边,容量为当前时间点的需求量 $D_{t,i}$。
\end{itemize}
\textbf{关键性质}:存在可行分配当且仅当该网络的最大流等于所有项目的总需求量 $\sum_{i=1}^n D_{t,i}$。

\subsection*{动态变化的本质}
不同时间点仅改变项目 $\rightarrow$ 汇点的边容量。直接对每个时间点单独跑最大流复杂度为 $O(q \cdot \text{maxflow})$,不可接受。

\subsection*{CDQ分治的核心思想}
利用\textbf{分治结构}处理时间序列,通过\textbf{增量更新}网络流状态,避免重复计算。在每一层分治中:
\begin{enumerate}[itemsep=0pt]
    \item 先递归处理左半区间的时间点。
    \item 将网络流状态更新到左区间结束时的状态(即中间时刻的状态)。
    \item 基于这个状态递归处理右半区间。
    \item 回退左区间的修改,恢复状态。
\end{enumerate}
这样,当递归到单个时间点时,当前流网络恰好对应该时刻的容量设置,可直接判断可行性。

\section*{解决方案}
\subsection*{1. 网络流建模与基本算法}
构造二分图 $G=(V,E)$:
\begin{align*}
    V &= \{S, T\} \cup \{R_1,\dots,R_m\} \cup \{P_1,\dots,P_n\} \\
    E &= \{(S,R_j) \mid j=1,\dots,m\} \\
      &\cup \{(R_j,P_i) \mid j \in S_i\} \\
      &\cup \{(P_i,T) \mid i=1,\dots,n\}
\end{align*}
边容量:
\begin{align*}
    c(S,R_j) &= C_j \\
    c(R_j,P_i) &= +\infty \\
    c(P_i,T) &= D_{t,i} \quad \text{(随时间变化)}
\end{align*}
使用Dinic算法求最大流。记 $V = n+m+2 \leq 102$,$E \leq n \cdot m + n + m \leq 2600$。

\subsection*{2. 支持容量增量更新的最大流}
需要支持快速修改一条边 $(P_i,T)$ 的容量,并调整当前流值。

\subsubsection*{容量增加}
设原容量为 $c_{\text{old}}$,新容量为 $c_{\text{new}} > c_{\text{old}}$。只需在残量网络上从 $P_i$ 出发寻找到 $T$ 的增广路,尝试增加流量 $\Delta = c_{\text{new}} - c_{\text{old}}$。

\subsubsection*{容量减少}
设原容量为 $c_{\text{old}}$,新容量为 $c_{\text{new}} < c_{\text{old}}$。如果当前通过该边的流量 $f \leq c_{\text{new}}$,则无需操作。否则,需要减少流量 $\Delta = f - c_{\text{new}}$:
\begin{enumerate}[itemsep=0pt]
    \item 将 $(P_i,T)$ 的容量暂时设为 $c_{\text{new}}$。
    \item 在残量网络上从 $T$ 出发,寻找到 $P_i$ 的路径(相当于反向增广),将多余的 $\Delta$ 流量“推回”到 $P_i$。
    \item 这个退流过程等价于在反向图上从 $T$ 到 $P_i$ 找增广路。
\end{enumerate}

\subsection*{3. CDQ分治框架}
\begin{algorithm}[H]
\caption{CDQ分治处理动态资源分配}
\begin{algorithmic}[1]
\Procedure{CDQ}{$l, r$}
    \If{$l = r$}
        \State 当前网络状态即为时间点 $l$ 的状态
        \State 计算最大流,判断是否等于 $\sum_i D_{l,i}$,记录答案
        \State \Return
    \EndIf
    \State $mid \gets \lfloor (l+r)/2 \rfloor$
    \State \Call{CDQ}{$l, mid$} \Comment{递归处理左区间}
    \State 记录当前网络流状态
    \For{$t = l$ to $mid$}
        \For{每个项目 $i$}
            \State 将边 $(P_i,T)$ 容量从 $D_{t,i}$ 更新为 $D_{t+1,i}$
        \EndFor
    \EndFor
    \State 现在网络处于时间点 $mid+1$ 的状态
    \State \Call{CDQ}{$mid+1, r$} \Comment{递归处理右区间}
    \State 回退上述所有容量更新,恢复进入本层时的状态
\EndProcedure
\end{algorithmic}
\end{algorithm}

\textbf{实现细节}:
\begin{itemize}[itemsep=0pt]
    \item 需要维护每条边 $(P_i,T)$ 的容量修改序列。
    \item 每次更新后立即调整当前流值(增广或退流)。
    \item 回退操作需要记录修改前的流量和容量,反向执行更新操作。
\end{itemize}

\subsection*{4. 整体算法流程}
\begin{enumerate}[itemsep=0pt]
    \item 读入数据,建立初始网络流图(项目到汇点容量设为 $D_{1,i}$)。
    \item 运行 Dinic 算法计算初始最大流(作为第1层分治的起点)。
    \item 调用 \textsc{CDQ}(1, q) 处理所有时间点。
    \item 输出每个时间点的可行性结果。
\end{enumerate}

\section*{复杂度分析}
\subsection*{时间复杂度}
\begin{itemize}[itemsep=0pt]
    \item \textbf{CDQ分治层数}:$O(\log q)$。
    \item \textbf{每层操作}:在区间 $[l,r]$ 中,需要应用左区间所有时间点的容量更新。每个时间点更新 $n$ 条边,共 $O(n \cdot (mid-l+1))$ 次更新。
    \item \textbf{单次更新代价}:容量增加/减少操作需要在残量网络上增广/退流。最坏情况下每次更新可能触发 $O(V \cdot E)$ 的Dinic算法,但实际中:
    \begin{itemize}[itemsep=0pt]
        \item 每次更新只影响一条边,增广/退流路径长度有限。
        \item 分治过程中网络流状态逐渐变化,不会每次都需要完全重算。
        \item 实际运行效率远低于理论上界。
    \end{itemize}
    \item \textbf{总复杂度}:$O(q \log q \cdot (n+m) \cdot E)$,在题目约束下可接受。
\end{itemize}

\subsection*{空间复杂度}
\begin{itemize}[itemsep=0pt]
    \item 网络流图:$O(V+E) = O(nm)$。
    \item 分治递归栈:$O(\log q)$。
    \item 修改记录:$O(q \cdot n)$ 用于回退操作。
    \item 总空间:$O(qn + nm)$。
\end{itemize}

\section*{总结}
本题结合了网络流建模、CDQ分治和动态图算法,核心在于:
\begin{enumerate}[itemsep=0pt]
    \item 将资源分配问题转化为二分图最大流问题。
    \item 利用CDQ分治处理时间序列,避免对每个时间点独立计算。
    \item 实现支持边容量增量更新的最大流算法,通过局部增广/退流高效调整流值。
\end{enumerate}
该解法充分利用了问题特性,通过分治结构将动态问题转化为静态问题序列,是处理一类在线/动态网络流问题的有效范式。

\end{document}