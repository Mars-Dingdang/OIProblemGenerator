\documentclass{article}
\usepackage{amsmath, amssymb, fullpage}
\begin{document}
\title{Editorial: Digital Divisibility Challenge}
\maketitle

\section*{Core Observation}
The problem requires counting substrings in a digit string that are divisible by $2^L$, where $L$ is the substring length. A critical insight is that for $L > 60$, $2^L$ exceeds the maximum value of any $L$-digit number (which is $10^L - 1$). Consequently, no $L$-digit number can be divisible by $2^L$ when $L > 60$. For $L \leq 60$, we efficiently compute substring values modulo $2^L$ using a sliding window with modular arithmetic, leveraging the property that $10^L \equiv 0 \pmod{2^L}$ (since $10^L = (2 \cdot 5)^L = 2^L \cdot 5^L$).

\section*{Solution}
We iterate over all valid substring lengths $L$ from 1 to $\min(n, 60)$, where $n$ is the input string length. For each $L$:
\begin{enumerate}
    \item \textbf{Initialization}: 
    \begin{itemize}
        \item If $L > n$, skip this $L$.
        \item Set $M = 2^L$ and $\text{mask} = M - 1$ (enabling fast modulo via bitwise AND).
        \item Initialize $\text{current} = 0$ to store the numeric value of the current window modulo $M$.
    \end{itemize}
    
    \item \textbf{First window computation}:
    Process the initial window $s[0..L-1]$:
    \[
    \text{for } j = 0 \text{ to } L-1: \quad \text{current} \gets (\text{current} \times 10 + (s[j] - \text{'0'})) \mathbin{\&} \text{mask}
    \]
    If $\text{current} = 0$, the substring is divisible by $2^L$; increment the count.
    
    \item \textbf{Sliding window updates}:
    For each subsequent window starting at index $i$ ($0 \leq i < n - L$):
    \begin{itemize}
        \item The window shifts from $s[i..i+L-1]$ to $s[i+1..i+L]$.
        \item Using $10^L \equiv 0 \pmod{M}$, the update simplifies to:
        \[
        \text{current} \gets (\text{current} \times 10 + (s[i+L] - \text{'0'})) \mathbin{\&} \text{mask}
        \]
        \item If $\text{current} = 0$, increment the count.
    \end{itemize}
\end{enumerate}
The total count is the sum of valid substrings across all $L \leq 60$. Substrings with $L > 60$ are skipped per the core observation.

\section*{Complexity Analysis}
\begin{itemize}
    \item \textbf{Time Complexity}: 
    The algorithm processes $L$ from 1 to 60 (constant). For each $L$, it scans the string once in $O(n)$ time (each window update is $O(1)$ due to bitwise operations). Total time is $O(60 \cdot n) = O(n)$.
    
    \item \textbf{Space Complexity}: 
    Only $O(1)$ auxiliary space is used per $L$ (storing $M$, mask, current, and loop variables). With $L$ bounded by 60, total space remains $O(1)$.
\end{itemize}
This approach efficiently handles the problem by eliminating $L > 60$ cases and using modular arithmetic for linear-time processing of small $L$.

\end{document}